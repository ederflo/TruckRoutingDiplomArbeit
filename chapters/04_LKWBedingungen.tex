Für die Erstellung von LKW-Routen müssen natürlich auch die vorhandenen Bedingungen,
wie zum Beispiel das maximal zugelassene Gesamtgewicht oder die Pausenregelungen berücksichtigt
werden.

    \section{LKW-Arten}
    Es gibt eine Vielzahl an unterschiedlichen LKW-Arten. Generell können diese Typen
    in fünf Klassen unterschieden werden.
    \begin{center}
        \resizebox{\textwidth}{!}{%
        \begin{tabular}{|l|l|l|}
            \hline
            \rowcolor[HTML]{9B9B9B} 
            \multicolumn{1}{|c|}{\cellcolor[HTML]{9B9B9B}{\color[HTML]{000000} Fahrzeugklasse}} &
            \multicolumn{1}{c|}{\cellcolor[HTML]{9B9B9B}{\color[HTML]{000000} zulässiges Gesamtgewicht}} &
            \multicolumn{1}{c|}{\cellcolor[HTML]{9B9B9B}{\color[HTML]{000000} Beispiel für Nutzfahrzeug}} \\ \hline
            N1 - Leichte Nutzfahrzeuge & bis 3,49 t         & Lieferfahrzeuge                     \\ \hline
            \rowcolor[HTML]{EFEFEF} 
            N1 - Leichte LKW           & 3,5 t bis 7,49 t   & Lieferung im Nahverkehr             \\ \hline
            N2 - Schwere LKW           & 7,5 t bis 11,49 t  & Lieferung im Regionalverkehr        \\ \hline
            \rowcolor[HTML]{EFEFEF} 
            N3 - Schwere LKW           & ab 12 t            & Baustellenverkehr, Güterfernverkehr \\ \hline
            N3 - Sattelzüge            & bis 40 t oder 44 t & Güterfernverkehr                    \\ \hline
        \end{tabular}%
        }
    \end{center}
    

    \section{Pausenregelung}

    \section{Fahrverbote}

    \section{Gesamtgewicht}