In diesem Kapitel stelle ich die verwendeten Technologien für mein Projekt vor.
\\
Die Anzahl an verschiedenen Möglichkeiten ein Projekt umzusetzen, ist in keiner 
Branche so hoch, wie in der Informatik. Es ist wichtig, von Beginn an Informationen 
über potenzielle Lösungen zu suchen und diese zu evaluieren.

\section{Backend}
Serverseitig ist die Logik implementiert. Die richtige Auswahl des Backends und deren Aufbau ist
enorm wichtig für einen reibungslosen Ablauf. Falschen Entscheidungen können enorme Folgen auslösen.
\\
Die Struktur des Backends war bereits vorhanden und auf dieser wurde aufgebaut.

    \subsection{ASP.NET Core}
    Active Server Pages .NET Core, kurz ASP.NET Core, ist der Nachfolger von ASP.NET. Dieses kostenlose
    Open-Source-Webframework bietet die Möglichkeit plattformunabhängige Applikationen zu entwickeln. 
    Es bietet eine modulare Software-Entwicklung, da eine Vielzahl von fertigen Paketen verschiedenster 
    Art zur Verfügung stehen.
    \\\\
    Da dieses Framework von Arduvi bereits verwendet wurde, ist mir die Entscheidung, welches
    Webframework für meine Diplomarbeit in Frage kommen würde, abgenommen worden.

    \subsection{Datenspeicherung}
    Die Datenverwaltung, wie auch die Datenspeicherung ist ein zentraler Bestandteil
    einer Applikation. Sie dient dazu, Inhalte permanent abzuspeichern, um diese jederzeit 
    wieder abrufen zu können. Es gibt verschiedene Methoden, wie Daten abgespeichert werden können.
    In meinem Fall habe ich den Cloud-Dienst Azure verwendet.

        \subsubsection{Azure Cosmos DB}
        Azure Cosmos DB ist ein global verteiltes, schema-freies und horizontal skalierbares 
        Datenbankservice aus dem Hause Microsoft. Das Datenbankservice ist Teil der Cloud-Computing-Plattform namens Azure.
        Mithilfe verschiedener APIs können Daten bestmöglich verwaltet werden.
        \\
        Da der Cloud-Dienst Azure komplett in die Applikation integriert ist, war es selbstverständlich,
        diesen auch zu nutzen.

        \subsubsection{Azure Cosmos DB SQL API}
        Mithilfe der SQL API können Daten dokumentenbasiert im Format JSON in die Azure Cosmos DB 
        eingetragen werden. Dies erfolgt in sogenannten Dokumenten. Weiteres können diese Dokumente
        in Partitionen unterteilt werden. Die einzelnen Einträge innerhalb eines Dokumentes werden 
        Items genannt. Dokumente werden in einem Container abgespeichert und Container wiederum in Datenbanken.
        \\
        Die Schnittstelle zwischen Applikation und Datenbank war bereits vorhanden und es mussten lediglich
        die Methoden für die Verwaltung der Routendaten implementiert werden.

        \subsubsection{Azure Cosmos DB Gremlin API}
        Die Gremlin API, welche auf die Azure Cosmos DB zugreift, wird verwendet um Graphen zu persistieren.
        Meist wird es benutzt, wenn die Beziehung zwischen Entitäten eine große Rolle spielt. Die sogenannten
        Property-Graphen bestehen aus folgenden Elementen:
        \begin{itemize}
            \item Knoten (Vertex)
            \item Kanten (Edge)
            \item Labels
            \item Eigenschaften (Properties)
        \end{itemize}
        Grundsätzlich besteht der Graph aus Vertices und Edges. Mithilfe des Labels wird ihnen ein Name zugeteilt.
        Dieser bestimmt den Typ eines Vertex bzw. einer Edge. Properties bestehen aus Key-Value-Pairs, welche
        einem Vertex oder einer Edge angehören. 
        \\
        Da die Schnittstelle zwischen Applikation und Datenbank noch nicht vorhanden war, musste diese
        komplett neu implementiert werden.

\section{Client}
Die Darstellung und Strukturierung der Daten wird clientseitig vollzogen. Dazu wurde der sogenannte 
"living standard" verwendet. Dieser beschreibt die Kombination aus HTML, JavaScript und CSS und wird
vom World Wide Web Consortium (W3C) ständig weiterentwickelt.

    \subsection{HTML}
    Die Hypertext Markup Language, kurz HTML, ist eine Auszeichnungssprache zum Darstellen und 
    Strukturieren von Elementen wie zum Beispiel Paragrafen, Listen, Tabellen oder auch Buttons.
    \\
    Sie ist die Sprache des World Wide Webs und wird von allen gängigen Browsern unterstützt.
    Allerdings dient HTML rein zur Strukturierung, denn die visuelle Darstellung wird mit CSS umgesetzt.

    \subsection{CSS}
    Cascading Style Sheets, kurz CSS, wird zur visuellen Gestaltung von HTML- oder XML-Code verwendet. 
    Damit sollte das Aussehen von der Strukturierung der Inhalte getrennt werden. Mittels CSS können
    zum Beispiel Farben angepasst, Animationen eingefügt oder auch Abstände zwischen Elementen
    eingestellt werden.

    \subsection{JavaScript}
    Die Skriptsprache JavaScript, kurz JS, welche dynamisch typisiert, objektorientiert und 
    klassenlos ist. Ursprünglich wurde sie im Jahr 1995 von Netscape entwickelt, um dynamische
    Websiten zu bauen. Mittlerweile allerdings kommt sie auch serverseitig oder für Microcontroller
    zum Einsatz. Objektorientiertes, prozedurales oder auch funktionales programmieren ist mit JavaScript
    möglich.

    \subsection{Razor Pages}
    Razor ist eine Markup-Syntax, welche die Möglichkeit bietet, serverseitigen Code in Websiten
    zu integrieren. Es ist somit keine Programmiersprache, sondern eine serverseitige Auszeichnungssprache.
    Üblicherweise wird .cshtml als Dateiendung verwendet. Bevor die Website an den Browser gesendet wird,
    wird der serverseitige Code (meist C\#) ausgeführt.

    \subsection{Template - Inspinia Admin Theme}
    Das Inspinia Admin Theme ist ein responsives Admin-Dashboard-Theme, das auf dem Bootstrap 4.x
    Framework aufbaut. Eine Vielzahl an Webframeworks und HTML-Templates wird somit bereits mitgeliefert.

    \subsection{Weitere Frameworks oder Libraries}
    Natürlich reichen diese fünf oben genannten Technologien aus, um eine voll fortschrittliche, voll
    funktionsfähige Website zu bauen. Es gibt allerdings eine vielzahl an JavaScript Libraries, welche
    einem viel Arbeit abnehmen und somit Zeit eingespart wird. Nun möchte ich einige der wichtigsten
    Bibliotheken für mein Projekt aufzählen.

    \subsubsection{Bootstrap}
    Bootstrap ist eines der beliebtesten open-source Frameworks für die Entwicklung von modernen 
    Webseiten mittels HTML, CSS und JS. Eines der wichtigsten Features sind Media-Queries, die für
    die responsive Darstellung eingesetzt werden können.

    \subsubsection{jQuery}
    jQuery ist eine umfangreiche JavaScript Library für einfacheres und schnelleres Zugreifen auf den
    HTML Dombaum. Viele JavaScript Frameworks bauen auf jQuery und somit ist es in einer Vielzahl
    von Projekten vertreten.