In diesem Kapitel stelle ich die verwendeten Technolgien und ihre Stärken, 
wie auch ihere Schwächen vor.
\\
Die Anzahl an verschiedenen Möglichkeiten ein Projekt umzusetzten ist in keiner 
Branche so hoch wie in der Informatik. Es ist wichtig, von Beginn an Informationen 
über potenzielle Lösungen zu suchen und diese zu evaluieren.

\section{Backend}
Server-Seitig ist die Logik implementiert. Die richtige Auswahl des Backends und deren Aufbau ist
enorm Wichtig für einen reibungslosen Ablauf. Falschen Entscheidungen können enorme Folgen auslösen.
\\
Die Struktur des Backends war bereits vorhanden und auf dieser wurde aufgebaut.

    \subsection{ASP .NET Core}
    ASP .Net Core ist der Nachfolger von ASP .NET.  Dieses kostenlose Open-Source-Webframework
    bietet die Möglichkeit platformunabhängige Applikationen zu entwickeln.
    \\
    Es bietet eine modulare Software-Entwicklung, da eine Vielzahl von fertigen Paketen
    verschiedenster Art zur Verfügung stehen.
    \\\\
    Da es Server-Seitig bereits von Arduvi verwendet wurde, ist mir die Entscheidung, welches
    Webframework für meine Diplomarbeit in Frage kommmen würde, abgenommen worden.


    \subsection{Datenspeicherung}
    Die Datenverwaltung, wie auch die Datenspeicherung ist ein zentraler Bestandteil
    einer Applikation.

        \subsubsection{Azure CosmosDB}
        a

        \subsubsection{Azure GraphDB}
        a

    \subsection{Weitere Frameworkds}
    a

\section{Client}
a

    \subsection{HTML}
    a

    \subsection{CSS}
    a

    \subsection{Javascript}
    a

    \subsection{Razor Pages}
    a

    \subsection{Weitere Frameworks}
    a   