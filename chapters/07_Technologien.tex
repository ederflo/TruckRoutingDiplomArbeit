In diesem Kapitel stelle ich die verwendeten Technolgien und ihre Stärken, 
wie auch ihere Schwächen vor.
\\
Die Anzahl an verschiedenen Möglichkeiten ein Projekt umzusetzten ist in keiner 
Branche so hoch, wie in der Informatik. Es ist wichtig, von Beginn an Informationen 
über potenzielle Lösungen zu suchen und diese zu evaluieren.

\section{Backend}
Server-Seitig ist die Logik implementiert. Die richtige Auswahl des Backends und deren Aufbau ist
enorm Wichtig für einen reibungslosen Ablauf. Falschen Entscheidungen können enorme Folgen auslösen.
\\
Die Struktur des Backends war bereits vorhanden und auf dieser wurde aufgebaut.

    \subsection{ASP .NET Core}
    ASP .Net Core ist der Nachfolger von ASP .NET.  Dieses kostenlose\\
    Open-Source-Webframework bietet die Möglichkeit platformunabhängige\\
    Applikationen zu entwickeln.
    \\
    Es bietet eine modulare Software-Entwicklung, da eine Vielzahl von fertigen Paketen
    verschiedenster Art zur Verfügung stehen.
    \\\\
    Da es Server-Seitig bereits von Arduvi verwendet wurde, ist mir die Entscheidung, welches
    Webframework für meine Diplomarbeit in Frage kommmen würde, abgenommen worden.

    \subsection{Datenspeicherung}
    Die Datenverwaltung, wie auch die Datenspeicherung ist ein zentraler Bestandteil
    einer Applikation. Sie dient dazu, Inhalte permanent abzuspeichern, um diese jederzeit 
    wieder abrufen zu können. Es gibt verschiedene Methoden, wie Daten abgespeichert werden können.
    In meinem Fall habe ich den Clound-Dienst Azure verwendet.

        \subsubsection{Azure Cosmos DB}
        Azure Cosmos DB ist ein global verteiltes, schema-freies und horizontal skalierbares 
        Datenbankservice aus dem Hause Microsoft. Das Datenbankservice ist Teil der Cloud-Computing-Plattform namens Azure.
        Mithilfe verschiedener APIs können Daten bestmöglich verwaltet werden.
        \\
        Da der Cloud-Dienst Azure komplett in die Applikation intigriert ist, war es selbstverständnis,
        diesen auch zu nutzen.

        \subsubsection{Azure Cosmos DB SQL API}
        Mithilfe der SQL API können Daten dokumentenbasiert im Format JSON in die Azure Cosmos DB 
        eingetragen werden. Dies erfolgt in sogenannten Dokumenten. Weiteres können diese Dokumente
        in Partitionen unterteilt werden. Die einzelnen Einträge innerhalb eines Dokumentes werden 
        Items genannt. Dokumente werden in einem Container abgespeichert und Container wiederum in Datenbanken.
        \\
        Die Schnittstelle zwischen Applikation und Datenbank war bereits vorhanden und es mussten jediglich
        die Methoden für die Verwaltung der Routendaten implementiert werden.

        \subsubsection{Azure Cosmos DB Gremlin API}
        Die Gremlin API, welche auf die Azure Cosmos DB zugreift, wird verwendet um Graphen zu persistieren.
        Meist wird es benutzt, wenn die Beziehung zwischen Entitäten eine große Rolle spielt. Die sogenannten
        Property-Graphen bestehen aus folgenden Elementen:
        \begin{itemize}
            \item Knoten (Vertex)
            \item Kanten (Edge)
            \item Labels
            \item Eigenschaften (Properties)
        \end{itemize}
        Grundsätzlich besteht der Graph aus Vertices und Edges. Mithilfe des Labels wird ihnen ein Name zugeteilt.
        Dieser bestimmt den Typ eines Vertex bzw. einer Edge. Properties bestehen aus Key-Value-Pairs, welche
        einem Vertex oder einer Edge angehören. 
        \\
        Da die Schnittstelle zwischen Applikation und Datenbank noch nicht vorhanden war, musste dise
        komplett neu implementiert werden.

    \subsection{Weitere Frameworkds}
    a

\section{Client}
Die Darstellung der Daten ist Client-Seitig. //todo: mehr schreiben

    \subsection{HTML}
    Die Hypertext Markup Language, kurz HTML, ist eine Auszeichnungssprache zum Strukturieren von
    

    \subsection{CSS}
    a

    \subsection{Javascript}
    a

    \subsection{Razor Pages}
    a

    \subsection{Weitere Frameworks}
    a   